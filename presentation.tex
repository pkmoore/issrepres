\documentclass[pdf]{beamer}
\usepackage{pifont}
\mode<presentation>{}
\title{Charting a Course Through Uncertain Environments: SEA Uses Past
Problems to Avoid Future Failures}
\author{Preston Moore, Justin Cappos, Phyllis Frankl, Thomas Wies}
\begin{document}


\begin{frame}
  \titlepage{}
\end{frame}


\begin{frame}{Deployment Bugs are Frequent and Expensive}
  \begin{itemize}
    \item{On this slide I should I mention the ClusterHQ/Oracle/Microsoft
      studies we talk about in the paper intro?}
    \item{A major factor in this is differing environments...}
  \end{itemize}
\end{frame}

\begin{frame}{How Does an Environment Influence an Application?}
  \begin{itemize}
  \item{Programs do not execute in isolation}
  \item{Instead, they operate:}
    \begin{itemize}
    \item{On the inputs they are provided}
    \item{\textit{In the context of the environment where they are executed}}
    \end{itemize}
    \item{Developers have become very good at testing applications by
        manipulating inputs}
      \begin{itemize}
        \item{But environments affect execution as well}
      \end{itemize}
  \end{itemize}
\end{frame}


\begin{frame}{What Does an Environment Provide to an Application?}
  \begin{itemize}
  \item{An implicit input in the form of environment variables, file
      contents, and other data accessible by the application}
  \item{An implementation of the executable resources on which an application depends}
    \begin{itemize}
    \item{e.g. libraries, callable OS functions}
    \end{itemize}
  \end{itemize}
\end{frame}


\begin{frame}{Environments Differ on Many Levels}
  \begin{itemize}
  \item{Versions of libraries and software dependencies}
  \item{Condition of the network on which the application relies}
  \item{Operating system states, which can vary}
    \begin{itemize}
    \item{\textit{e.g.\ filesystem state, environment variable values}}
    \end{itemize}
  \end{itemize}
\end{frame}


\begin{frame}{Environmental Variations Can Reveal Bugs!}
  An application may fail if:
  \begin{itemize}
  \item{A file it needs is missing}
  \item{It can not handle unusual filesystem configurations}
  \item{It does not handle possible network performance issues}

    % \item{Weird environmental conditions may be uncommon so developers do not
    %   know that they need to handle them!}
      \end{itemize}
      Failure to deal with these conditions results in incorrect behavior --- i.e.\ bugs.

      These bugs are hard to identify because setting up different environments is
      expensive.
\end{frame}


\begin{frame}{What is needed...}
  \begin{itemize}
    \item{Is a methodical way to record, preserve, and test against
      problematic environmental features}
  \end{itemize}
\end{frame}


% \begin{frame}{The Typical Deployment Scenario For A New Version of an Application}
%   \begin{itemize}
%   \item{Applications are tested under ideal conditions}
%   \item{Bugs do not present themselves until an application has
%       been deployed to other, less controlled environments}
%   \item{Bugs must then be diagnosed and fixes must be created and distributed}
%   \item{What is needed is a way to test how applications react to particular
%       environmental conditions \textbf{BEFORE} they are deployed}
%   \end{itemize}
% \end{frame}


\begin{frame}{Simulating Environmental Anomalies (SEA)}
  \begin{itemize}
  \item{SEA allows its users to:}
    \begin{itemize}
    \item{Generically capture and encode unusual environmental
      conditions (referred to as ``anomalies'')}
    \item{Expose a running application to simulated anomalies}
    \item{Assess whether or not the application may be responding this
      simulation}
    \end{itemize}
  \end{itemize}
\end{frame}


\begin{frame}{The Key Ideas}
  \begin{itemize}
  \item{Problematic environmental properties can often be detected in an
    application's}
    \begin{itemize}
    \item{function calls}
    \item{system calls}
    \item{other interactions}
  \end{itemize}
  \item{Exposing an application to these properties gives an idea of how
    likely it is to fail once deployed}
  \end{itemize}
\end{frame}


\begin{frame}{SEA on the Half-Shell}
  \begin{enumerate}
  \item{Record an application as it fails in an anomalous environment}
  \item{Analyze the recording to identify what aspects of the environment
    (known as an anomaly) caused the failure}
  \item{Simulate the anomaly in the presence of another application}
  \item{Assess how the application responds}
  \end{enumerate}
\end{frame}


\begin{frame}{Generic Example}
  \begin{enumerate}
    \item{An application runs fails an environment because problematic
      feature $X$ is present}
    \item{$X$ is analyzed to determine how it causes environmental
      interactions to differ (we call this difference $\Delta X$}
    \item{$\Delta X$ is used to construct a mutator \textit{mutX()} and
      a checker \textit{checkX()}}
    \item{\textit{mutX()} is used to mutate the system calls of other
      applications in order to simulate the presence of $X$}
    \item{\textit{checkX()} is used to evaluate the application's response
      to $X$}
  \end{enumerate}
\end{frame}


\begin{frame}{An Concrete Example - Simulating the ``Absence'' of a File}
  \begin{itemize}
  \item{Consider an application that depends on reading from file F}
  \end{itemize}

  Normal open: open(F) = 3

  File is missing: open(F) = -1 ENOENT

  \begin{itemize}
  \item{In this case:}
    \begin{itemize}
      \item{$X$ is the fact that F is missing}
      \item{$\Delta X$ is changing the result of open to -1 ENOENT}
      \item{\textit{mutX()} modifies the result of open() calls of another
        application to return -1 ENOENT}
      \item{\textit{checkX()} determines whether the application has
        recognized the failing return value}
    \end{itemize}
  \end{itemize}
\end{frame}


\begin{frame}{Identifying and Encoding Anomalies}
  \begin{enumerate}
  \item{Content}
  \end{enumerate}
\end{frame}


\begin{frame}{Mutating System Call Results}
  \begin{itemize}
  \item{Record trace of application under test}
  \item{Perturb trace so it now contains anomalous system call results}
    \begin{itemize}
    \item{Some mutations can be done programmatically}
    \item{Some mutations require manual effort (at this time)}
    \end{itemize}
  \item{Replay execution of application using mutated trace}
  \end{itemize}
\end{frame}


% \begin{frame}{Replaying a System Call: The Ideal Case}
%   \begin{itemize}
%   \item{Intercept the system call and reproduce its results and side effects}
%     \begin{itemize}
%     \item{All ``simple'' system calls}
%     \item{All system calls involving file descriptors that do not really exist}
%       \begin{itemize}
%       \item{That is, file descriptors produced from system calls that we
%           previously intercepted and handled}
%       \end{itemize}
%     \end{itemize}
%   \item{Convert system call into a call to getpid()}
%   \item{Parse corresponding line from system call trace}
%   \item{Update register values and memory contents to reflect the results we
%       \textbf{parsed out of the trace}}
%   \end{itemize}
% \end{frame}
%
%
% \begin{frame}{Replaying a System Call: When The Ideal Case Does Not Work}
%   \begin{itemize}
%   \item{The effects of some system calls are too complex to deal with}
%     \begin{itemize}
%     \item{e.g.\ mmap()}
%     \end{itemize}
%   \item{In this case we allow them to be executed natively by:}
%     \begin{itemize}
%       \item{Instructing ptrace to allow the application to execute until the
%           next system call}
%       \item{Checking upcoming system call to ensure it matches the corresponding
%           line from the trace being replayed}
%       \item{Restarting execution of the application, thereby allowing it to
%           actually make a real system call}
%       \item{Checking the results to ensure they, too, match the results from the
%           corresponding line in the trace being replayed}
%     \end{itemize}
%   \end{itemize}
% \end{frame}
%
%
% \begin{frame}{Replaying a System Call Trace}
%   \begin{itemize}
%   \item{Execute application}
%   \item{Use Linux's Ptrace facilities to interrupt execution at every system
%       call}
%   \item{Adjust program state to match contents of system call trace}
%   \item{Because replay process ensures system calls respond the same way as they
%       did when the trace was recorded, the application will follow the same
%       execution}
%   \end{itemize}
% \end{frame}


\begin{frame}{Assessing the Response of an Application}
  \begin{itemize}
  \item{If an application changes its system call behavior, the replay diverges,
      i.e. execution no longer matches the trace}
  \item{This means the application may be trying to handle the anomalous behavior}
    \begin{itemize}
    \item{\textit{Possibly correct behavior}}
    \end{itemize}
  \item{If an application does not change its system call behavior, replay
      continues}
    \begin{itemize}
    \item{\textit{Application has not made an effort to handle anomalous conditions}}
    \end{itemize}
  \end{itemize}
\end{frame}


\begin{frame}{CrashSimulator: A Proof of Concept SEA Implementation}

  Implementation Building Blocks:

  \begin{itemize}
  \item{32-bit Linux}
  \item{Modified version rr record-and-replay debugger}
  \item{Python Process Supervisor}
  \item{C Python Module}
    \begin{itemize}
    \item{Manipulation of System Calls using Ptrace}
    \item{Structuring Data}
    \end{itemize}
  \end{itemize}
\end{frame}


\begin{frame}{CrashSimulator's Architecture}
    Diagram from paper here
\end{frame}


\begin{frame}{Anomalies Used}
  \begin{itemize}
  \item{Unexpected Linux File Types}
  \item{Cross-disk File Movement}
  \item{Extremely Long Network Timeouts}
  \item{Additional Features:}
    \begin{itemize}
    \item{Null Mutator}
    \item{Default Checker}
    \end{itemize}
  \end{itemize}
\end{frame}


\begin{frame}{Unexpected Linux File Types}
  \begin{itemize}
  \item{Linux provides different types of files with different capabilities}
  \item{Application should confirm file type before processing a file}
  \item{Otherwise, problems:}
    \begin{itemize}
    \item{Applications hang on FIFOs files}
    \item{Applications consume disk space/memory/etc. processing ``infinitely
        large'' files}
    \end{itemize}
  \item{Found bugs in coreutils utilities and gnupg}
  \end{itemize}
\end{frame}


\begin{frame}{Unusual File Type Bugs}
  Include discussion and new results table here
  % Mention aspell bug in particular
\end{frame}


\begin{frame}{Cross-Disk File Movement - The Takeaway}
  \begin{itemize}
    \item{Applications fail to diligently check file types before operating
      on files leading to failures}
  \end{itemize}
\end{frame}


\begin{frame}{Cross-Disk File Movement Discussion}
  \begin{itemize}
  \item{Moving a file around on the same disk is trivial: rename() system
      call}
  \item{This system call does not support moving a file from one disk to
      another}
  \item{An application must detect this situation and either fail
      gracefully or perform the file move ``manually''}
  \item{Otherwise, problems:}
    \begin{itemize}
    \item{Loss of file contents}
    \item{Loss of file metadata}
    \item{Race conditions}
    \item{Filling of disk space}
    \end{itemize}
  \item{Race condition in Python's shutil}
  \item{Loss of extended file attributes in Rust's std::fs}
  \item{Fill disk space bug in C++'s Boost::filesystem}
  \end{itemize}
\end{frame}


\begin{frame}{Cross-Disk File Movement Results}
  Include table here
\end{frame}


\begin{frame}{Extremely Long Network Timeouts Discussion}
  \begin{itemize}
  \item{Applications that do not configure reasonable timeouts are vulnerable
      to a malicious communication participant dragging communication out over a
      long period of time}
  \item{This process ties up resources and greatly lowers the effort required to
      consume all available resources, preventing legitimate access}
  \item{``Slowloris'' Denial of Service attack against http servers}
    \begin{itemize}
    \item{involves keeping a connection open to a vulnerable server by
        introducing just-below-the-timeout delays between sending HTTP headers}
        \end{itemize}
  \item{MANY Python HTTP/HTTPS/web modules vulnerable}
  \end{itemize}
\end{frame}


\begin{frame}{Cross-Disk File Movement - The Takeaway}
  \begin{itemize}
    \item{Well known, popular applications are not moving files correctly}
  \end{itemize}
\end{frame}


\begin{frame}{Extremely Long Network Timeouts Results}
  Include results table here
\end{frame}


\begin{frame}{Extremely Long Network Timeouts - The Takeaway}
  \begin{itemize}
    \item{Major applications and libraries are vulnerable to maliciously
      timed communications}
  \end{itemize}
\end{frame}


\begin{frame}{In Conclusion}
  \begin{itemize}
  \item{An application's environment is an often-neglected cause of bugs}
  \item{These bugs are difficult to find pre-deployment because replicating
      unusual environments is expensive}
  \item{CrashSimulator provides a technique for simulating these unusual
      environments in order to identify bugs before deployment, thereby
      saving time and expense}
  \item{In tests, CrashSimulator has found these types of environmental bugs in
      both popular applications and libraries}
  \end{itemize}
\end{frame}

\end{document}
